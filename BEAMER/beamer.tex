\documentclass{beamer}
\usetheme{CambridgeUS}

\setbeamertemplate{caption}[numbered]{}

\usepackage{enumitem}
\usepackage{tfrupee}
\usepackage{amsmath}
\usepackage{amssymb}
\usepackage{gensymb}
\usepackage{graphicx}
\usepackage{txfonts}

\def\inputGnumericTable{}

\usepackage[latin1]{inputenc}                                 
\usepackage{color}                                            
\usepackage{array}                                            
\usepackage{longtable}                                        
\usepackage{calc}                                             
\usepackage{multirow}                                         
\usepackage{hhline}                                           
\usepackage{ifthen}
\usepackage{caption} 
\captionsetup[table]{skip=3pt}  
\providecommand{\pr}[1]{\ensuremath{\Pr\left(#1\right)}}
\providecommand{\cbrak}[1]{\ensuremath{\left\{#1\right\}}}
\renewcommand{\thefigure}{\arabic{table}}
\renewcommand{\thetable}{\arabic{table}}                                     
                               
\title{AI1110 \\ Assignment 8}
\author{U.S.M.M TEJA \\ CS21BTECH11059}
\date{17th May 2022}


\begin{document}
	% The title page
	\begin{frame}
		\titlepage
	\end{frame}
	
	% The table of contents
	\begin{frame}{Outline}
    		\tableofcontents
	\end{frame}
	
	% The question
	\section{Question}
	\begin{frame}{question 2.20}
A player tosses a penny from a distange onto the surface of a square table ruled in 1 in.
squares. If the penny is $\frac{3}{4}$ in. in diameter, what is the probability that it will fall entirely
inside a square (assuming that the penny lands on the table).
	\end{frame}
	
	% The solution
	\section{Solution}
	\begin{frame}{Solution}
here as long as the centre is r units away from the sides of the square it will stay inside the square and that is our question requirement.
so there's a small square of side length d−2r in each square where the center can fall without the coin extending beyond the grid square.
    \end{frame}

\begin{frame}{calculations}
 \begin{align}
    & =\frac{(1-2r)^2}{1^2} & \\
    & = (1 - \frac{3}{4})^2& \\
    & = \frac{1}{16}&
\end{align}   
\end{frame}

\end{document}
